\documentclass[12pt,a4paper]{article}
\usepackage[utf8]{inputenc}
\usepackage{amsmath}
\usepackage{amsfonts}
\usepackage{amssymb}
\usepackage{graphicx}
\usepackage[left=3.00cm, right=3.00cm]{geometry}
\author{Pascal Granier, William Cibille}
\title{DSS-4 Lab report\\
Keystroke audio recognition}
\date{Mittwoch 25. Juni 2014}

\begin{document}

\maketitle

\begin{abstract}
blablabla
\end{abstract}

\tableofcontents
\newpage

\section*{Introduction}


\subsection{Motivation}

This project involves
\begin{itemize}
\item Signal analysis, especially for Acoustics
\item Machine learning
\end{itemize}
The aim of the project is to show it is possible to recognize the text somebody types on the keyboard. 

All keyboards, even silent ones make noise when you type. According to our research \cite{Zhuang}, these sound emanations are due to the way keyboards are built. The material is not assembled uniformly so the keyboard is irregular. Thus, when you type a key, it will sound different from the next one. 

In our work, we will choose strong experimental conditions in order to implement a first version of our algorithm. We obtained a fully working code from our masters. And we had to improve it. 

\subsection{Research task}

\subsection{Our development axis}

\section{How we managed to improve the results}

\subsection{Recording conditions}

% établir la liste des conditions pour que notre programme fonctionne 
% par exemple: ambiance silencieuse, un seul clavier, position des micros, etc

\subsection{Stereo-recording solution}


\bibliographystyle{plain}
\bibliography{dss4-pg-wc_report}
\end{document}